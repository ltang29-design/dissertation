\documentclass[12pt]{article}
\usepackage[utf8]{inputenc}
\usepackage[english]{babel}
\usepackage[margin=1in]{geometry}
\usepackage{setspace}
\usepackage{amsmath}
\usepackage{graphicx}
\usepackage[authoryear,round]{natbib}
\usepackage{hyperref}
\usepackage{booktabs}
\usepackage{longtable}
\usepackage{array}
\usepackage{multirow}

\doublespacing
\title{\textbf{VR Navigation Task: Detailed Description}}
\date{}

\begin{document}
\maketitle

\section{Virtual Reality Navigation Task: Complete Specification}

\subsection{Task Overview}

Participants navigated a virtual reality maze with assistance from an AI agent. The task was designed to create genuine trust-reliance scenarios where participants must choose whether to follow or ignore agent recommendations, with real consequences for their choices.

\subsubsection{Task Objectives}

\textbf{Explicit objective} (told to participants): ``Navigate through the maze and return to the exit as quickly and accurately as possible with assistance from the AI agent.''

\textbf{Implicit objective} (research purpose): Create opportunities for:
\begin{itemize}
    \item Trust development through agent interaction
    \item Trust calibration through agent errors
    \item Behavioral trust measurement through compliance choices
    \item Decision confidence measurement through decision times
    \item Trust repair observation through error acknowledgment
\end{itemize}

\subsection{Virtual Environment Specifications}

\subsubsection{Maze Architecture}

\textbf{Overall dimensions}: 50m $\times$ 50m virtual space (perceived as approximately half a city block)

\textbf{Wall specifications}:
\begin{itemize}
    \item Height: 3.0 meters (prevents seeing over walls)
    \item Thickness: 0.3 meters
    \item Texture: Brick pattern with subtle variations (aids depth perception)
    \item Color: Neutral gray-brown (reduces visual fatigue)
\end{itemize}

\textbf{Corridor width}: 2.5 meters (comfortable for avatar + participant presence)

\textbf{Ceiling}: Open sky with ambient daylight (no enclosed feeling, reduces claustrophobia)

\textbf{Floor specifications}:
\begin{itemize}
    \item Base texture: Stone tile pattern
    \item Grid overlay: 10cm $\times$ 10cm subtle lines (aids proprioception and distance judgment)
    \item Color variation: Slight variations to prevent monotony
    \item Non-reflective: Prevents visual artifacts
\end{itemize}

\subsubsection{Lighting and Atmosphere}

\textbf{Lighting system}:
\begin{itemize}
    \item Primary: Simulated overhead sunlight (consistent across maze)
    \item Intensity: 800 lux equivalent (comfortable indoor-outdoor level)
    \item Shadow: Soft shadows cast by avatar and walls (enhances depth perception)
    \item No flicker: Steady illumination prevents discomfort
\end{itemize}

\textbf{Atmospheric effects}:
\begin{itemize}
    \item Sky: Clear blue with subtle cloud movement
    \item Ambient sound: Gentle white noise at 40 dB (masks external sounds)
    \item No music: Prevents emotional priming
    \item Temperature indicator: None (focused on navigation)
\end{itemize}

\subsubsection{Landmark Objects (15 Total)}

Landmarks placed throughout maze to serve as:
\begin{enumerate}
    \item Visual reference points for navigation
    \item Memory cues for +MAPK agent references
    \item Environmental richness (prevents sterile environment)
\end{enumerate}

\begin{longtable}{p{0.15\textwidth}p{0.25\textwidth}p{0.15\textwidth}p{0.35\textwidth}}
\caption{Complete Landmark Specifications} \\
\toprule
\textbf{Corner} & \textbf{Landmark} & \textbf{Position} & \textbf{Description} \\
\midrule
\endfirsthead
\multicolumn{4}{c}{\tablename\ \thetable\ -- \textit{Continued from previous page}} \\
\toprule
\textbf{Corner} & \textbf{Landmark} & \textbf{Position} & \textbf{Description} \\
\midrule
\endhead
\midrule \multicolumn{4}{r}{\textit{Continued on next page}} \\
\endfoot
\bottomrule
\endlastfoot

1 & Red upholstered chair & Right side, 2m from corner & Armchair, 0.9m tall, distinct red color \\
2 & Blue fabric sofa & Left side, 3m before corner & 2-seater, navy blue, 0.8m tall \\
3 & Wooden dining table & Center of corridor, 5m before corner & Brown oak, 4 chairs, 1.2m $\times$ 2.0m \\
4 & Large cardboard box & Right corner, blocking partial path & 1m $\times$ 1m $\times$ 1m, brown corrugated \\
5 & Potted plant & Left side, at corner & Ficus tree, 1.5m tall, green ceramic pot \\
Between 5-6 & Small bookshelf & Right wall midpoint & 1.2m tall, books visible, brown wood \\
6 & Green landscape painting & Wall ahead, 4m before corner & Framed art, 1m $\times$ 0.7m, mountain scene \\
7 & Standing lamp & Right corner & Floor lamp, 1.8m tall, cream shade, turned on \\
8 & Small wooden crate & Left side, 1m before corner & 0.5m cube, unstained wood, open top \\
Between 8-9 & Fire extinguisher & Wall mount, right side & Red cylinder, 0.6m tall, realistic detail \\
9 & Yellow storage box & Center, 3m before corner & Plastic container, 0.4m tall, bright yellow \\
10 (Exit) & Red chair (same as Corner 1) & Right side at exit & Same chair, signaling return \\
Additional & Welcome mat & At starting position & ``START'' text, green color \\
Additional & Exit door & End position & Large green door, ``EXIT'' sign \\
Additional & Directional signs & At 3 corners & Subtle arrows, partially hidden \\
\end{longtable}

\textbf{Landmark placement rationale}:
\begin{itemize}
    \item \textbf{Distinctive}: Each landmark visually unique, easy to remember
    \item \textbf{Realistic}: Everyday objects participants would recognize
    \item \textbf{Non-distracting}: Objects static, not animated (prevents distraction)
    \item \textbf{Scale appropriate}: Human-scale objects (aids immersion)
\end{itemize}

\subsection{Navigation Mechanics and Controls}

\subsubsection{Movement System}

\textbf{Locomotion type}: Continuous smooth locomotion (not teleportation)
\begin{itemize}
    \item \textbf{Speed}: Fixed at 1.5 m/s virtual speed (equivalent to brisk walking pace)
    \item \textbf{Direction}: Forward only (no backward movement possible)
    \item \textbf{Turning}: Executed via controller joystick or keyboard
    \item \textbf{Stopping}: Not permitted---continuous forward momentum (creates task urgency)
\end{itemize}

\textbf{Rationale for continuous movement}:
\begin{enumerate}
    \item Creates time pressure, making agent guidance valuable
    \item Prevents excessive deliberation (forces decisions)
    \item Mimics real-world navigation constraints (can't stand still indefinitely)
    \item Reduces VR sickness (smooth continuous movement better than stop-start)
\end{enumerate}

\subsubsection{Control Scheme}

\textbf{Input device}: [Oculus Touch controllers / Xbox controller / Keyboard]

\textbf{Control mapping}:
\begin{itemize}
    \item \textbf{Direction selection} (at decision points):
    \begin{itemize}
        \item Button A / Arrow Left: Turn left
        \item Button B / Arrow Right: Turn right
        \item Button X / Arrow Up: Continue forward
    \end{itemize}
    
    \item \textbf{Help request}:
    \begin{itemize}
        \item Button Y / Spacebar: Request additional help from agent
        \item Help count displayed: ``Help available: X/5''
    \end{itemize}
    
    \item \textbf{Emergency controls}:
    \begin{itemize}
        \item Trigger / Escape: Pause task
        \item Grip / P: Report discomfort (immediate headset removal)
    \end{itemize}
\end{itemize}

\textbf{Control tutorials}: Participants practiced each control during 5-minute tutorial before main task.

\subsubsection{Visual Display at Decision Points}

At each of 10 corners, participants encountered structured decision interface:

\textbf{Decision prompt display}:
\begin{enumerate}
    \item \textbf{Header}: ``Decision Point X of 10'' (semi-transparent overlay, top-center)
    \item \textbf{Available directions}: Visual icons showing possible paths
    \begin{itemize}
        \item Left arrow: If left path available
        \item Forward arrow: If forward path available
        \item Right arrow: If right path available
    \end{itemize}
    \item \textbf{Agent recommendation}: Highlighted direction (glowing border)
    \item \textbf{Help counter}: ``Help available: X/5'' (bottom-right)
    \item \textbf{Timer}: Decision time counter (visible to researchers only, not participants)
\end{enumerate}

\textbf{Interface appearance}: Semi-transparent UI overlay (doesn't block view), modern minimalist design, high contrast for readability.

\subsection{AI Agent Design and Behavior}

\subsubsection{Avatar Visual Design}

\textbf{Anthropomorphic features}:
\begin{itemize}
    \item \textbf{Form}: Humanoid shape (head, torso, arms, legs)
    \item \textbf{Height}: 1.7 meters (average human height)
    \item \textbf{Proportions}: Realistic human proportions
    \item \textbf{Appearance}: Stylized rather than photorealistic (uncanny valley avoidance)
    \item \textbf{Gender presentation}: Androgynous/neutral (no gendered features)
    \item \textbf{Clothing}: Simple casual outfit (blue shirt, gray pants)
    \item \textbf{Face}: Simplified features (eyes, nose, mouth visible but stylized)
    \item \textbf{Animation}: Smooth natural movement (motion-captured walking animation)
\end{itemize}

\textbf{Color scheme}:
\begin{itemize}
    \item Skin tone: Medium neutral (not resembling any specific ethnicity)
    \item Clothing: Cool colors (blue, gray) to contrast with warm-colored landmarks
    \item Highlighting: Subtle glow effect when speaking
\end{itemize}

\subsubsection{Agent Positioning and Movement}

\textbf{Spatial relationship to participant}:
\begin{itemize}
    \item \textbf{Distance}: 1.0 meter ahead of participant (constant)
    \item \textbf{Orientation}: Always facing same direction as participant
    \item \textbf{Height}: Eye level approximately matching participant's view
    \item \textbf{Lateral position}: Centered in participant's field of view
\end{itemize}

\textbf{Movement synchronization}:
\begin{itemize}
    \item Agent moves when participant moves (locked synchrony)
    \item Agent turns when participant turns (immediate response)
    \item Walking animation: Natural stride matching movement speed
    \item No independent movement: Agent never walks away or toward participant
\end{itemize}

\textbf{Personality-specific movement differences}:
\begin{itemize}
    \item \textbf{Introvert Agent}: 
    \begin{itemize}
        \item Movement: Smooth, steady, minimal arm swing
        \item Posture: Upright, contained
        \item Head movement: Minimal, only when pointing
    \end{itemize}
    
    \item \textbf{Extrovert Agent}:
    \begin{itemize}
        \item Movement: Slightly bouncy, energetic arm swing
        \item Posture: Open, slightly forward-leaning
        \item Head movement: More frequent, looks around environment
    \end{itemize}
\end{itemize}

\subsubsection{Agent Behavior at Decision Points}

When participant approaches corner (3 meters before decision point):

\textbf{Sequence of events}:
\begin{enumerate}
    \item \textbf{T - 3 seconds}: Agent slows slightly, turns head toward upcoming decision
    \item \textbf{T - 1 second}: Agent raises right arm, points toward recommended direction
    \item \textbf{T = 0}: Decision prompt appears on screen
    \item \textbf{T + 0.5s}: Agent provides verbal recommendation
    \item \textbf{T + 1s to decision}: Agent maintains pointing gesture
    \item \textbf{After decision}: Agent lowers arm, resumes normal walking
\end{enumerate}

\textbf{Pointing gesture details}:
\begin{itemize}
    \item \textbf{Introvert Agent}: 
    \begin{itemize}
        \item Arm extension: 30° from body (subtle gesture)
        \item Duration: Holds for 1.5 seconds
        \item Speed: Slow arm raise (1.0s to reach position)
        \item Finger: Index finger extended, others curled
    \end{itemize}
    
    \item \textbf{Extrovert Agent}:
    \begin{itemize}
        \item Arm extension: 60° from body (emphatic gesture)
        \item Duration: Holds for 0.8 seconds, repeats if decision time $>$ 5s
        \item Speed: Quick arm raise (0.5s to reach position)
        \item Hand: Full hand pointing, more animated
    \end{itemize}
\end{itemize}

\subsection{Task Phases and Progression}

\subsubsection{Phase 1: Guided Navigation (Corners 1--5)}

\textbf{Duration}: 12--15 minutes average (\textit{SD} = 3 minutes)

\textbf{Description}: Initial maze exploration with agent guidance. Participants navigate from START position through 5 decision points.

\textbf{Cognitive demands}:
\begin{itemize}
    \item Spatial orientation in unfamiliar environment
    \item Trust formation based on early agent performance
    \item Learning maze layout
    \item Integrating agent verbal and gestural cues
\end{itemize}

\textbf{Memory manipulation}: Not yet active; +MAPK and $-$MAPK conditions receive similar guidance (memory references begin in Phase 2 where they're most natural for return navigation).

\textbf{Agent accuracy in Phase 1}: 80\% (4 correct, 1 error at Corner 3)

---

\subsubsection{Phase 1 to Phase 2 Transition}

\textbf{Trigger}: Participant completes Corner 5 decision

\textbf{Visual transition}:
\begin{enumerate}
    \item Screen fades slightly (3-second fade to 50\% brightness)
    \item Text overlay appears: ``Excellent progress! You've reached the halfway point.''
    \item 2-second pause
    \item New instruction: ``Now navigate back to the starting point where the EXIT is located. Continue following the agent's guidance.''
    \item 3-second pause (participant reads)
    \item Screen fades back to full brightness
\end{enumerate}

\textbf{Agent transition dialogue}:
\begin{itemize}
    \item \textbf{Introvert Agent}: ``We have completed the forward navigation. We will now return to the exit point. I will continue to provide guidance for the return journey.''
    
    \item \textbf{Extrovert Agent}: ``Great job so far! We've made it halfway! Now let's head back to where we started---I'll keep helping you navigate back to the exit. Ready? Let's go!''
\end{itemize}

---

\subsubsection{Phase 2: Return Navigation (Corners 6--10)}

\textbf{Duration}: 15--20 minutes average (\textit{SD} = 4 minutes)

\textbf{Description}: Return journey to starting point/exit. Participants retrace path (though not always exact reverse---some corners offer different return routes).

\textbf{Cognitive demands}:
\begin{itemize}
    \item Memory of previous path (increased cognitive load)
    \item Trust maintenance or revision based on accumulated evidence
    \item Comparison of agent memory references (+MAPK) to own memory
    \item Decision confidence under increased uncertainty
\end{itemize}

\textbf{Memory manipulation ACTIVE}: +MAPK agents make 5--7 explicit memory references; $-$MAPK agents provide equivalent dialogue without memory content.

\textbf{Agent accuracy in Phase 2}: 60\% (3 correct, 2 errors at Corners 7 and 9)

\textbf{Rationale for lower Phase 2 accuracy}: Tests trust maintenance when performance deteriorates; mimics real-world AI performance degradation in complex scenarios.

\subsection{Complete Decision Point Specifications}

\begin{longtable}{p{0.08\textwidth}p{0.08\textwidth}p{0.15\textwidth}p{0.15\textwidth}p{0.44\textwidth}}
\caption{Detailed Specifications for All 10 Decision Points} \\
\toprule
\textbf{Corner} & \textbf{Phase} & \textbf{Correct Path} & \textbf{Agent Rec.} & \textbf{Agent Dialogue} \\
\midrule
\endfirsthead
\multicolumn{5}{c}{\tablename\ \thetable\ -- \textit{Continued from previous page}} \\
\toprule
\textbf{Corner} & \textbf{Phase} & \textbf{Correct Path} & \textbf{Agent Rec.} & \textbf{Agent Dialogue} \\
\midrule
\endhead
\midrule \multicolumn{5}{r}{\textit{Continued on next page}} \\
\endfoot
\bottomrule
\endlastfoot

\multirow{2}{*}{1} & \multirow{2}{*}{1} & \multirow{2}{*}{Right} & \multirow{2}{*}{Right ✓} & \textbf{Introvert}: ``I recommend we turn right here. This direction appears optimal.'' \\
& & & & \textbf{Extrovert}: ``Let's go right! I've got a good feeling about this one!'' \\
\midrule

\multirow{2}{*}{2} & \multirow{2}{*}{1} & \multirow{2}{*}{Forward} & \multirow{2}{*}{Forward ✓} & \textbf{Introvert}: ``Continue forward. This path should lead us correctly.'' \\
& & & & \textbf{Extrovert}: ``Straight ahead---let's keep going, we're doing great!'' \\
\midrule

\multirow{2}{*}{3} & \multirow{2}{*}{1} & \multirow{2}{*}{Right} & \multirow{2}{*}{Left ✗} & \textbf{Introvert}: ``I believe left is the appropriate direction here.'' \\
& & & & \textbf{Extrovert}: ``Okay, let's try left---yeah, left feels right!'' \\
& & & & \textit{[ERROR - leads to dead end]} \\
& & & & \textbf{Error acknowledgment (Introvert)}: ``I apologize for the error. We should backtrack and try the right path.'' \\
& & & & \textbf{Error acknowledgment (Extrovert)}: ``Oops, my bad! Let's go back and try right instead.'' \\
\midrule

\multirow{2}{*}{4} & \multirow{2}{*}{1} & \multirow{2}{*}{Right} & \multirow{2}{*}{Right ✓} & \textbf{Introvert}: ``Right appears to be the correct choice at this junction.'' \\
& & & & \textbf{Extrovert}: ``Right turn coming up! This should be good!'' \\
& & & & \textbf{Memory (+MAPK)}: ``We had that wrong turn earlier, but let's go right this time.'' \\
\midrule

\multirow{2}{*}{5} & \multirow{2}{*}{1} & \multirow{2}{*}{Forward} & \multirow{2}{*}{Forward ✓} & \textbf{Introvert}: ``We should proceed straight ahead from here.'' \\
& & & & \textbf{Extrovert}: ``Forward! We're making excellent progress!'' \\
& & & & \textit{[End of Phase 1 - Transition message appears]} \\
\midrule

\multirow{2}{*}{6} & \multirow{2}{*}{2} & \multirow{2}{*}{Forward} & \multirow{2}{*}{Forward ✓} & \textbf{Introvert}: ``For our return journey, continue forward.'' \\
& & & & \textbf{Extrovert}: ``Alright, let's head forward on our way back!'' \\
& & & & \textbf{Memory (+MAPK)}: ``Do you remember this area? We came through here earlier. Let's go straight.'' \\
\midrule

\multirow{2}{*}{7} & \multirow{2}{*}{2} & \multirow{2}{*}{Left} & \multirow{2}{*}{Forward ✗} & \textbf{Introvert}: ``I believe forward is the appropriate direction for returning.'' \\
& & & & \textbf{Extrovert}: ``Let's keep going forward---should get us back!'' \\
& & & & \textbf{Memory (+MAPK)}: ``Based on what I remember, forward should work here.'' \\
& & & & \textit{[ERROR - second agent error]} \\
& & & & \textbf{Error acknowledgment (Introvert)}: ``That was incorrect. We need to go left instead.'' \\
& & & & \textbf{Error acknowledgment (Extrovert)}: ``Whoops, my bad again! Left is the way to go.'' \\
\midrule

\multirow{2}{*}{8} & \multirow{2}{*}{2} & \multirow{2}{*}{Left} & \multirow{2}{*}{Left ✓} & \textbf{Introvert}: ``Turn left at this point.'' \\
& & & & \textbf{Extrovert}: ``Left turn here! Almost there!'' \\
& & & & \textbf{Memory (+MAPK)}: ``I remember that lamp over there---let's go left toward it.'' \\
\midrule

\multirow{2}{*}{9} & \multirow{2}{*}{2} & \multirow{2}{*}{Forward} & \multirow{2}{*}{Left ✗} & \textbf{Introvert}: ``I recommend turning left here.'' \\
& & & & \textbf{Extrovert}: ``Let's go left---we're nearly back!'' \\
& & & & \textbf{Memory (+MAPK)}: ``If I recall correctly, left should bring us back.'' \\
& & & & \textit{[ERROR - third and final agent error]} \\
& & & & \textbf{Error acknowledgment (Introvert)}: ``I was mistaken. Forward is correct.'' \\
& & & & \textbf{Error acknowledgment (Extrovert)}: ``Oops! Forward is actually the way!'' \\
\midrule

\multirow{2}{*}{10} & \multirow{2}{*}{2} & \multirow{2}{*}{Left} & \multirow{2}{*}{Left ✓} & \textbf{Introvert}: ``Turn left here. The exit should be visible shortly.'' \\
& & & & \textbf{Extrovert}: ``Last turn! Left here and we're done!'' \\
& & & & \textbf{Memory (+MAPK)}: ``I recognize this---it's the red chair from the beginning! Left to the exit!'' \\
& & & & \textit{[Exit door becomes visible after turn]} \\
\endlongtable}

\subsection{Help System Mechanics}

\subsubsection{Help Request Procedure}

When participant presses Help button:

\textbf{1st Help Request}:
\begin{itemize}
    \item Agent repeats recommendation with additional reasoning
    \item \textbf{Introvert}: ``I recommend [direction] because [reasoning].''
    \item \textbf{Extrovert}: ``Yeah, definitely [direction]! Here's why...''
    \item Duration: 5--7 seconds additional dialogue
\end{itemize}

\textbf{2nd Help Request}:
\begin{itemize}
    \item Agent points more emphatically (larger gesture)
    \item Provides landmark reference
    \item \textbf{Example}: ``See that [landmark]? We should go toward it.''
\end{itemize}

\textbf{3rd--5th Help Requests}:
\begin{itemize}
    \item Agent reiterates with increasing confidence
    \item \textbf{3rd}: ``I'm quite sure [direction] is best.''
    \item \textbf{4th}: ``I'm very confident about [direction].''
    \item \textbf{5th}: ``I strongly believe [direction] is the correct choice.''
\end{itemize}

\textbf{Limit}: Maximum 5 help requests per corner (prevents infinite loops)

\textbf{Help exhaustion}: If participant requests 6th help, message appears: ``No additional help available for this decision. Please make your choice.''

\subsubsection{Help Request Metrics}

For each help request, system logged:
\begin{itemize}
    \item Corner number (1--10)
    \item Help request number (1--5)
    \item Time since decision prompt (seconds)
    \item Agent's accuracy at that corner (correct/incorrect)
    \item Participant's ultimate decision (after help)
\end{itemize}

This enabled calculation of help-seeking patterns, overreliance, underreliance, and misplaced reliance metrics.

\subsection{Outcome Feedback System}

\subsubsection{Immediate Feedback}

Participants received immediate environmental feedback for their choices:

\textbf{Correct Decision}:
\begin{itemize}
    \item Path opens ahead (no obstruction)
    \item Participant continues forward smoothly
    \item Next landmark visible in distance
    \item Agent provides positive reinforcement:
    \begin{itemize}
        \item \textbf{Introvert}: ``That was correct. We are making progress.''
        \item \textbf{Extrovert}: ``Yes! Good choice! Onward!''
    \end{itemize}
\end{itemize}

\textbf{Incorrect Decision}:
\begin{itemize}
    \item Participant advances 5--10 meters
    \item Dead end appears (solid wall ahead)
    \item Screen message: ``This path doesn't lead to the exit. Returning to decision point.''
    \item Automatic backtrack (participant transported back to corner)
    \item \textbf{Cost}: Time penalty (~15 seconds) + need to rechoose direction
    \item Agent acknowledges error if it was agent's recommended path
\end{itemize}

\subsubsection{Performance Tracking (Researcher View Only)}

Experimenter monitor displayed real-time:
\begin{itemize}
    \item Current corner (X/10)
    \item Decisions made (total count including backtracks)
    \item Errors so far (participant errors + agent-led errors)
    \item Help requests used (X/50 total available)
    \item Elapsed time (minutes:seconds)
    \item Participant position in maze (overhead map view)
\end{itemize}

\textbf{Participants did NOT see}:
\begin{itemize}
    \item Their accuracy percentage
    \item Agent's accuracy percentage  
    \item Performance comparisons
    \item Time limits (task was untimed from participant perspective)
\end{itemize}

\textbf{Rationale}: Avoided meta-cognitive awareness that might artificially improve calibration.

\subsection{Memory Function Manipulation: Complete Dialogue Scripts}

\subsubsection{+MAPK Memory References by Corner}

\begin{longtable}{p{0.12\textwidth}p{0.8\textwidth}}
\caption{Complete Memory Reference Dialogue (+MAPK Conditions)} \\
\toprule
\textbf{Corner} & \textbf{Memory Dialogue (Example)} \\
\midrule
\endfirsthead
\toprule
\textbf{Corner} & \textbf{Memory Dialogue} \\
\midrule
\endhead
\bottomrule
\endlastfoot

4 & \textbf{Introvert}: ``You may recall we encountered difficulty at the previous corner. Right should be correct here.'' \\
& \textbf{Extrovert}: ``Remember that wrong turn we took? Let's go right this time!'' \\
\midrule

6 & \textbf{Introvert}: ``This appears to be the area we traversed earlier. Continue forward based on our prior experience.'' \\
& \textbf{Extrovert}: ``Hey, we've been here before! Let's keep going straight!'' \\
\midrule

7 & \textbf{Introvert}: ``Referencing our earlier path, I believe forward is the appropriate direction for our return.'' \\
& \textbf{Extrovert}: ``Based on where we've been, forward should get us back!'' \\
& \textit{[Note: This memory reference is INCORRECT---demonstrates memory can mislead]} \\
\midrule

8 & \textbf{Introvert}: ``I recall observing that lamp previously. Left toward the lamp should be correct.'' \\
& \textbf{Extrovert}: ``I remember that lamp! Let's head left toward it!'' \\
\midrule

9 & \textbf{Introvert}: ``Consulting my memory of our route, left appears to lead back to our starting point.'' \\
& \textbf{Extrovert}: ``If I remember right, left should take us back!'' \\
& \textit{[Note: Memory reference INCORRECT again]} \\
\midrule

10 & \textbf{Introvert}: ``I recognize this location. This is the red chair from our initial starting area. Left will return us to the exit.'' \\
& \textbf{Extrovert}: ``That's the chair from the start! We're basically back! Left to the exit!'' \\
\endlongtable}

\textbf{Key design feature}: Memory references in +MAPK were sometimes associated with incorrect recommendations (Corners 7, 9), testing whether memory creates false confidence even when guidance is wrong.

\subsubsection{$-$MAPK Non-Memory Dialogue}

\begin{longtable}{p{0.12\textwidth}p{0.8\textwidth}}
\caption{Non-Memory Dialogue ($-$MAPK Conditions)} \\
\toprule
\textbf{Corner} & \textbf{Non-Memory Dialogue (Example)} \\
\midrule
\endfirsthead
\toprule
\textbf{Corner} & \textbf{Non-Memory Dialogue} \\
\midrule
\endhead
\bottomrule
\endlastfoot

4 & \textbf{Introvert}: ``Right appears to be the correct direction at this junction.'' \\
& \textbf{Extrovert}: ``Let's go right here---looks good!'' \\
\midrule

6 & \textbf{Introvert}: ``Continue forward from this point.'' \\
& \textbf{Extrovert}: ``Keep going straight! We're making great progress!'' \\
\midrule

7 & \textbf{Introvert}: ``I believe forward is the appropriate choice for returning.'' \\
& \textbf{Extrovert}: ``Forward should do it! Let's keep moving!'' \\
\midrule

8 & \textbf{Introvert}: ``Turn left at this corner.'' \\
& \textbf{Extrovert}: ``Left turn! Almost there!'' \\
\midrule

9 & \textbf{Introvert}: ``Left is my recommendation for this decision.'' \\
& \textbf{Extrovert}: ``Let's go left! We're nearly back!'' \\
\midrule

10 & \textbf{Introvert}: ``Turn left here. The exit should be visible shortly.'' \\
& \textbf{Extrovert}: ``Last turn! Left and we're done!'' \\
\endlongtable}

\textbf{Content balancing}: All $-$MAPK dialogue matched +MAPK on:
\begin{itemize}
    \item Word count per utterance ($\pm$5 words)
    \item Utterance count per corner
    \item Information content (direction + confidence/reasoning)
    \item Personality markers (formal vs. casual maintained)
\end{itemize}

\textbf{Key difference}: References only current moment, no past references.

\subsection{Decision Recording and Data Collection}

\subsubsection{Automated Data Logging}

For each decision point, system automatically recorded:

\textbf{Pre-decision data}:
\begin{itemize}
    \item Corner number (1--10)
    \item Phase (1 or 2)
    \item Timestamp (millisecond precision)
    \item Participant position (x, y, z coordinates)
    \item Agent recommendation (Left/Right/Forward)
    \item Correct path (Left/Right/Forward)
\end{itemize}

\textbf{During-decision data}:
\begin{itemize}
    \item Decision time start (when prompt appears)
    \item Help requests (count and timing of each)
    \item Help dialogue delivered (which help messages given)
    \item Decision time ongoing (updated每millisecond)
\end{itemize}

\textbf{Post-decision data}:
\begin{itemize}
    \item Participant choice (Left/Right/Forward selected)
    \item Decision time final (milliseconds from prompt to selection)
    \item Accuracy (1 if correct path chosen, 0 if incorrect)
    \item Compliance (1 if followed agent, 0 if didn't)
    \item Outcome (success: continued forward; error: backtrack required)
    \item Backtrack count (if error led to dead end)
\end{itemize}

\textbf{Additional decisions data}:
\begin{itemize}
    \item If backtrack occurred: Participant made second decision at same corner
    \item Up to 5 decisions possible per corner (initial + 4 backtracks)
    \item All decisions logged but only \textit{first decision} analyzed for compliance
\end{itemize}

\subsubsection{Data Quality Checks}

\textbf{Real-time validation}:
\begin{itemize}
    \item Decision time $>$ 60s triggered warning (possible distraction)
    \item Help exhaustion (5 requests) flagged for review
    \item Excessive backtracks ($>$3 at one corner) triggered experimenter check
\end{itemize}

\textbf{Post-session validation}:
\begin{itemize}
    \item All 10 corners completed (verified)
    \item All timestamps sequential (no time reversals)
    \item All decisions within plausible ranges
    \item No data corruption or system errors
\end{itemize}

\subsection{Task Completion and Success Criteria}

\subsubsection{Successful Completion}

\textbf{Completion defined as}:
\begin{enumerate}
    \item All 10 decision points navigated
    \item Exit door reached
    \item Task not terminated early (no VR sickness, equipment failure)
\end{enumerate}

\textbf{Completion rate}: 95.8\% (88/92 final sample completed successfully)

\textbf{Completion time}:
\begin{itemize}
    \item Mean: 27.3 minutes (\textit{SD} = 6.8)
    \item Range: 18--45 minutes
    \item Includes decision times + movement time + any backtracks
\end{itemize}

\subsubsection{Task Success Metrics (Research Perspective)}

From research perspective, ``success'' was defined not by completion speed but by:
\begin{enumerate}
    \item \textbf{Data completeness}: All measures captured (100\% of completers)
    \item \textbf{Manipulation exposure}: Participant experienced all experimental features
    \item \textbf{Engagement}: Participant made genuine decisions (not random button pressing)
    \item \textbf{Comprehension}: Post-task questions indicated understanding of task
\end{enumerate}

All 92 final participants met these criteria.

\subsection{Pilot Testing and Task Validation}

\subsubsection{Pilot Study Details}

\textbf{Pilot sample}: \textit{n} = 12 participants (not included in main study)

\textbf{Pilot objectives}:
\begin{enumerate}
    \item Validate maze difficulty (challenging but solvable)
    \item Test agent dialogue naturalness
    \item Confirm personality manipulation detectability
    \item Verify equipment comfort (VR sickness rates)
    \item Estimate session duration
    \item Test data collection systems
\end{enumerate}

\textbf{Pilot findings led to adjustments}:
\begin{itemize}
    \item Reduced maze complexity (originally 12 corners, reduced to 10)
    \item Adjusted agent speech rate (originally too fast for comprehension)
    \item Added mid-task comfort checks
    \item Refined memory dialogue for naturalness
    \item Increased help limit from 3 to 5 per corner
\end{itemize}

\subsection{Summary: Task Design Rationale}

The VR navigation task was designed to:
\begin{enumerate}
    \item Create \textbf{genuine reliance opportunities} (moderate difficulty requiring guidance)
    \item Enable \textbf{trust calibration assessment} (70\% agent accuracy with identifiable errors)
    \item Provide \textbf{behavioral trust measurement} (compliance choices have real consequences)
    \item Allow \textbf{trust dynamics examination} (two phases enable within-subjects comparison)
    \item Ensure \textbf{ecological validity} (navigation is real-world relevant task)
    \item Maintain \textbf{experimental control} (VR enables standardization impossible in real navigation)
\end{enumerate}

The task successfully achieved all design objectives, yielding rich multi-dimensional trust data enabling comprehensive analysis of memory, personality, and matching effects.

\bibliography{references}
\bibliographystyle{apa}

\end{document}




